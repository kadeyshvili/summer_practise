\documentclass{beamer}
\usepackage[utf8]{inputenc}

\usepackage{tikz}
\usetheme{Berkeley}
\usecolortheme{default}

%Russian-specific packages
%--------------------------------------
\usepackage[T2A]{fontenc}
\usepackage[utf8]{inputenc}
\usepackage[russian]{babel}
%--------------------------------------

\hypersetup{
    colorlinks=true,
    linkcolor=blue,
    filecolor=magenta,      
    urlcolor=cyan,
    pdftitle={Overleaf Example},
    pdfpagemode=FullScreen,
    }
%------------------------------------------------------------
%This block of code defines the information to appear in the
%Title page
\title[About Beamer] 
{Presentation for LaTex course}

\author[Kadeyshvili] % (optional)
{Polina Kadeyshvili\inst{1}}

\date[LTXC 2022] % (optional)
{LaTeX Course, July 2022}

\logo{\includegraphics[height=1cm]{images/logo.jpg}}


\AtBeginSection[]
{
  \begin{frame}
    \frametitle{Table of Contents}
    \tableofcontents[currentsection]
  \end{frame}
}

\begin{document}

\frame{\titlepage}

\begin{frame}
\frametitle{Table of Contents}
\tableofcontents
\end{frame}


\section{Introduction}

\begin{frame}
\frametitle{Introduction}

\begin{itemize}
    \item In this presentation I will recreate one of the slides form our course in linear algebre and geometry

\end{itemize}
\end{frame}






\section{Критерий диагонализуемости линейного оператора}
\begin{frame}
\frametitle{Критерий}
\begin{center}
    V - векторное простронство над полем F, dimV = n $\phi : V \xrightarrow{} V $ - линейный оператор
\end{center}
\begin{block}{Теорема(критерий диагонализуемости)}
$\phi$ 
\text диагонализуем тогда и только тогда, когда одновременно выполнены следующие условия:\\
\begin{enumerate}  
	\item $\chi(t)$ - разлагается на линейные множители над F
	\item геоментрическая мультипликативность равна алгебраической $\forall \lambda \in Spec\phi$
\end{enumerate}
\end{block}
\end{frame}



\section{Conclusion}
\begin{frame}
\frametitle{Conclusion}
\begin{center}
    Thank you for your attention!
\end{center}
\end{frame}
\end{document}